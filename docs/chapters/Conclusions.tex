%% ----------------------------------------------------------------
%% Conclusions.tex
%% ---------------------------------------------------------------- 
\chapter{Conclusions and Reflection} \label{Chapter: Conclusions}
% 对于普通用户而言,本系统实现了用户自主注册,自主进行清洁能源电力购买并自主将清洁能源电力证书兑换为清洁能源积分的功能,同时用户可以用清洁能源电力积分兑换礼品市场中的礼品,另外,用户的清洁能源电力积分也可以用来相互交易。用户的功能需求和安全需求以及性能需求都符合需求文档的描述。对于管理员而言,可以随时监视系统的订单,用户信息,符合用户故事中对管理员的描述。对于发电终端而言,本系统设计了可以自动颁发证书的智能合约,因此也完成了需求文档的要求。值得提出的是系统在误操作和实用性方面仍有较大可提升空间,距离实际的投入消费领域还有很大差距,需要细化需求,精细开发。
% 需要指出的是,因为气候变化以及能源问题严峻,本项目所提出的碳排放市场普及化以及碳排放市场与其他领域挂钩是一个很有可能被实现的设想,因此本项目很可能未来拥有很高的社会以及商业价值
For general users, the system implements the function that users can independently register, independently make clean power purchases and independently redeem clean power certificates for clean power points, while users can exchange clean power points for gifts in the gift market, in addition, the clean power points of users can also be used to trade with each other. The functional and security requirements and performance requirements of the users are in accordance with the requirements document. For administrators, the system can be monitored at any time for orders, user information, in line with the description of administrators in the user story. For the power generator terminal, the system is designed with a smart contract that can automatically issue certificates, so it also fulfills the requirements of the requirement document. It is worth to propose that the system still has a large space for improvement in terms of misuse and practicality, and there is still a big gap from the actual input to the consumer field, which needs to refine the requirements and fine development.

It is important to note that the idea of universalizing the carbon market and linking it to other fields, as proposed in this project, is a very likely one to be realized because of the seriousness of climate change and energy issues, and therefore this project is likely to have a high social and commercial value in the future.
\section{Future Work}
% 与区块链相同,人工智能近些年也与许多传统领域开始开始了逐步的结合,并且效果不凡,因此未来可以将人工智能技术与本项目进行结合,开发出一个基于区块链的智能清洁电力和碳排放交易系统,使用算法博弈论方法为系统设计智能代理,为交易系统构建一个市场价格机制,使得交易系统的所有利益相关者获得最优收益,同时利用机器学习或强化学习算法分析用户数据,对用户进行画像从而提高用户体验;对于区块链网络底层,可以使用负载均衡技术,在分布式网络运行成本和用户交易速度之间取得最优平衡;对于前后端开发,可以使用Golang或者Java Spring框架替换NodeJS来实现高并发系统;对于拓展性而言,未来可以将系统开发成标的物可插拔的形式,即可以在需要的时候将诸如公共交通票据,自行车行程等项目方便快速的接入碳排放市场。总之,在这个领域,有很多方向都有很大的可能性。
Similar to blockchain, artificial intelligence has also started to gradually combine with many traditional fields in recent years, and the effect is extraordinary. Therefore, artificial intelligence technology can be combined with this project in the future to develop an intelligent clean power and carbon emission trading system based on blockchain, using algorithmic game theory methods to design intelligent agents for the system and build a market price mechanism for the trading system, so that an optimal revenue can be obtained by all stakeholders of the trading system. At the same time, machine learning or reinforcement learning algorithms can be used to analyze user data and profile users to improve user experience; for the underlying blockchain network, load balancing technology can be used to achieve an optimal balance between distributed network operation cost and user transaction speed; for front and back-end development, Golang or Java Spring framework instead of NodeJS to achieve a highly concurrent system; for expandability, the system can be developed into a pluggable form of the target object in the future, which means that items such as public transportation tickets and bicycle trips can be easily and quickly connected to the carbon emission market when needed. In short, there are many directions with great possibilities in this field.